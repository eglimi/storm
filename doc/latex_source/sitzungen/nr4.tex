\section{Sitzung 4 vom 12. November 2003}


%-----
\subsection{Teilnehmer}
H. Huser (hhu), M. Egli (meg), M. Winiger (mwi)


%-----
\subsection{Themen}
	\begin{itemize}
		\item Lebensdauer von Objekten
		\item Doku
		\item Planung/Fortschritt
	\end{itemize}


%-----
\subsection{Lebensdauer von Objekten}
Es m�ssen verschiedene Fragen bez�glich Lebensdauer der Objekte angeschaut 
werden.

Ist es besser wenn die Objekte nach einem Commit zerst�rt werden, 
oder k�nnen sie evtl. sp�ter wieder verwendet werden? Wird die Konsitenz 
bei Objekten aus der IdentitiyMap �berpr�ft? K�nnen die Objekte eingeteilt 
werden in eher statische Daten und solche, die man jedesmal neu von der 
Datenbank holen muss? Soll die Wahl zwischen Geschwindigkeit und Aktualit�t 
dem Benutzer �berlassen werden?

%-----
\subsection{Doku}
Wir haben die beiden Technologien XML Schema und Reflection im Vergleich 
dokumentiert. Wir m�ssen den Teil noch �berarbeiten und werden ihn vor der 
n�chsten Sitzung mailen.

%-----
\subsection{Planung/Fortschritt}
CodeSmith bereitet uns immer wieder Schwierigkeiten, weil vieles nicht 
gerade auf Anhieb funktioniert. Wir haben etwas fr�her mit der 
Implementation einer neuen HsrOrderApp begonnen und sind jetzt an den 
ersten Versuchen die Templates einzubauen.