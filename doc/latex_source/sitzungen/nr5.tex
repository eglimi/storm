\section{Sitzung 5 vom 18. November 2003}


%-----
\subsection{Teilnehmer}
H. Huser (hhu), M. Egli (meg), M. Winiger (mwi)


%-----
\subsection{Themen}
	\begin{itemize}
		\item Generischer/generativer Ansatz
		\item Enums, Lookup-Tables
	\end{itemize}


%-----
\subsection{Generischer/generativer Ansatz}

Es zeigt sich, dass der generative Ansatz viel aufwendiger 
zu realisieren ist, als am Anfang angenommen. Es gibt sehr 
viele Spezialf�lle. Eine Applikation wie die HsrOrderApp 
ist zu klein, dass sich der ernsthafte Einsatz von 
Code-Generierung lohnen w�rde.


%-----
\subsection{Enums, Lookup-Tables}

Enums sind einer dieser Spezialf�lle. Es gibt verschiedene 
M�glichkeiten diese auf die Datenbank zu mappen. Eine 
M�glichkeit w�ren Lookup-Tables. Dabei werden die 
Enum-Beschreibungen, mit eindeutigen IDs, in einzelnen oder 
einer gemeinsamen Tabelle abgelegt.

Eine entscheidende Rolle spielt f�r den Mapping-Code, wie 
die Enums in der Datenbank abgelegt sind. Liegt ein 
unver�nderbares Datenbank-Schema vor, muss der Code mit 
grosser Wahrscheinlichkeit spezifisch geschrieben werden.

Abgesehen davon werden die Enums in der HsrOderApp-Datenbank 
nicht sehr sch�n abgelegt. Sie sind nicht vollst�ndig 
normalisiert und somit redundant als VARCHAR abgelegt, was 
das mapping noch ein St�ck komplizierter macht
