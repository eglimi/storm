\section{Sitzung 3 vom 3. November 2003}


%-----
\subsection{Teilnehmer}
H. Huser (hhu), M. Egli (meg), M. Winiger (mwi)


%-----
\subsection{Themen}
	\begin{itemize}
		\item Anforderungs-Spezifikation
		\item Technischer Bericht
		\item Weiteres Vorgehen
	\end{itemize}


%-----
\subsection{Anforderungs-Spezifikation}

\noindent Q: Es gibt unsererseits Unklarheiten bez�glich dem Verfassen einer 
Anforderungs-Spezifikation.

\noindent A: Vor allem im ersten Teil der Arbeit geht es darum die bestehende 
Applikation zu analysieren. Wir m�ssen L�sungsans�tze suchen 
und herausfinden was die beste L�sung ist. Darum werden sich 
Anforderungen erst im Verlaufe des Projekt ergeben.
	
%-----
\subsection{Technischer Bericht}

Der erste Teil des technischen Berichts liegt vor. Es handelt sich 
dabei um eine Zusammenfassung der f�r unser Projekt wichtigen 
Pattern aus dem Buch von Fowler \cite{Fowler03} und einer 
kurzen Beschreibung zu CodeSmith \cite{CodeSmith}.

Wir werden das PDF auf \url{http://ooxgen.linda.homelinux.net/} 
laden und w�rden gerne an der n�chsten Sitzung ein Feedback erhalten.


%-----
\subsection{Weiteres Vorgehen}

Die Analyse der bestehenden Applikation, die Erstellung von Templates	
und das Dokumentieren der Erkenntnisse erfolgt im Moment fast 
gleichzeitig. Anfangs Woche werden wir den Schwerpunkt auf die 
Spezifikation der XML Schemas und das programmieren des zum parsen 
ben�tigten Code setzen. 

