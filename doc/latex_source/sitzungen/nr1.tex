\section{Sitzung 1 vom 20. Oktober 2003}

\subsection{Teilnehmer}
H. Huser (hhu), M. Egli (meg), M. Winiger (mwi)

\subsection{Themen}
	\begin{itemize}
		\item Besprechung Aufgabenstellung
		\item Allgemeine Fragen zum Projektstart
	\end{itemize}
\subsection{Aufgabenstellung}
	Die Aufgabenstellung ist noch nicht fix. Es ist m�glich, dass im Verlauf der Arbeit noch Verschiebungen der
	erwarteten Resultate m�glich sind.

\subsection{Entscheidungen}
\begin{itemize}
	\item Der Projektplan muss versioniert werden, damit �nderungen w�hrend dem Projekt auch im
	Nachhinein ersichtlich sind.
	\item Wir setzen eine Online-Zeiterfassung ein. Die Beschreibung zu den Eintr�gen dient
	gleichzeitig als Projekt-Tagebuch.\\
	\url{http://homer.skymx.net/zeit/ooxgen/}
	\item Wir w�rden die Dokumentation gerne mit \LaTeX{} erstellen. Wir versuchen eine L�sung zu finden, dass wir
	die Dokumentation, bzw. einzelne Dokumente davon, in ein zu Microsoft Word kompatibles Format umwandeln k�nnen.
	\item Einzelne Dokumente (z.B. Technischer Bericht) werden in Englisch verfasst.
\end{itemize}

\subsection{Hinweise}
	\begin{itemize}
		\item (hhu): Auf der MSDN Seite finden wir Pattern von Microsoft, die f�r uns n�tzlich sein
		k�nnten.\\
		\url{http://msdn.microsoft.com/architecture/patterns/MSpatterns/}\\
		\url{http://msdn.microsoft.com/architecture/application/default.aspx?pull=/library/en-us/dnbda/html/distapp.asp}
		\item (hhu): Der erwartete Einsatz f�r die Diplomarbeit liegt bei 50h/Woche.
	\end{itemize}
	
\subsection{Ziele}
\begin{itemize}
	\item Erstellen des Projektplans, damit dieser an der n�chsten Sitzung besprochen werden kann.
	\item Einarbeiten in HsrOrderApp\_S3
	\item Vertiefung von Pattern und .NET Technologien
\end{itemize}
