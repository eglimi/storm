\section{Sitzung 2 vom 28. Oktober 2003}

\subsection{Teilnehmer}
H. Huser (hhu), M. Egli (meg), M. Winiger (mwi)

\subsection{Themen}
	\begin{itemize}
	  \item Wochenr�ckblick
		\item Projektplan v1.0
		\item Dokumentation
	\end{itemize}

\subsection{Wochenr�ckblick}
\begin{itemize}
  \item Der Projektplan (v0.1) liegt vor und die Meilensteine sind in der Doku
    beschrieben.
  \item Die Technologiestudien Patterns und CodeSmith sind gr�sstenteils
    abgeschlossen.
\end{itemize}

\subsection{Projektplan v0.1}
Der Projektplan ist soweit in Ordnung. Die Planung ist f�r uns nicht einfach, da
wir uns zuerst mit den Technologien und Patterns vertraut machen m�ssen. Wichtig
ist, dass wir auftretende �nderungen am Projektplan sofort mit (hhu) besprechen.

\subsection{Dokumentation}
	\begin{description}
		\item[Risikoanalyse]
		  Zu allgemein. Konkretere L�sungen m�ssen angegeben werden.
		\item[Meilenstein 3]
		  Das Ziel ist zu hoch angesetzt. Es reicht wenn wir zeigen
		  welche Art von Mappings durch generativen Code ersetzt werden k�nnen.
		\item[Implementation]
		  Die Erstellung der Templates, welche als Werkzeuge
		  eingesetzt werden k�nnen, sollten in der Planung von der Implementation der
		  Beispiel-Applikation (HsrOrderApp\_S4) getrennt werden.
		\item[Bibliographie]
		  Die MSDN Architecture Patterns von Microsoft sollten in die
		  Bibliographie aufgenommen werden.
		\item[Allgemein]
		  Wir generieren aus dem Technical Report ein eigenst�ndiges
		  Dokument.
	\end{description}
