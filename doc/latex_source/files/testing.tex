\chapter{Ausgangslage}
	
	Getestet wurde die Funktionalit�t der HsrOrderApp\_S4. Da diese Applikation alle
	unterst�tzten F�lle der Code Generierung abdeckt, ist damit auch die Richtigkeit
	der Templates gew�hrleistet.

	\section{Testumgebung}
		Die Test wurden lokal ausgef�hrt und damit ist die Testumgebung dieselbe, wie
		in Kapitel \ref{cha:ressourcen} in der Tabelle \ref{tab:projektMgmtResources}
		angegeben wurde.
	
	\section{Datenbank}
		Das Design der Datenbank wurde vorgegeben. Es handelt sich dabei um die
		Datenbank f�r die HsrOrderApp. Diese wurde f�r die in dieser Arbeit
		erstellten Referenzimplementation verwendet und dient auch als Datenbank f�r 
		die Tests.

\chapter{HsrOrderApp\_S4}
	Wir haben entschieden, die bestehenden Tests der HsrOrderApp\_S3 zu erweitern. 
	Dazu haben wir Facade, WebServices, TestProxy und TestApplikation �bernommen 
	und an unsere Architektur angepasst.

	Getestet wurde damit der generierte Code der Domain Object und der Mapper.
	
	\section{Anforderungen an die Applikation}
	\begin{itemize}
		\item Neue Objekte
				
			Objekte �ber die Factory erstellen und in der UnitOfWork als new markieren.
		
		\item Objekte �ndern
		
			Geladene Objekte ver�ndern und in der UnitOfWork als dirty markieren.
		
		\item Objekte in die Datenbank schreiben
		
			Neue Objekte in die Datenbank einf�gen und ge�nderte Objekte in der Datenbank 
			auf den neusten Stand bringen.
		
		\item Objekte aus der Datenbank lesen
		
			Die Daten f�r neu zu erstellende Objekte werden mittels find Methoden aus der 
			Datenbank geholt.
		
		\item Objekte als gel�scht markieren
		
			Objekte f�r das l�schen in der Datenbank bei der UnitOfWork registrieren.
		
		\item Rekursives L�schen
		
			Wird ein Objekt mit ToMany Relationen gel�scht, m�ssen alle zugeordneten 
			Objekte vorher auch gel�scht werden.
				
		\item Werfen von Concurrency Exceptions
		
			Werden versucht veraltete Objekte auf die Datenbank zu schreiben, muss eine 
			Concurrency Exception geworfen werden.
		
	\end{itemize}
	
	\section{Durchgef�hrte Tests}
		Die folgenden drei Test werden mehrmals hintereinander ausgef�hrt. Zuerst werden 
		jeweils die in der letzen Iteration erstellten Objekte gesucht und wenn vorhanden 
		gel�scht. Objekte einer ToMany Relation m�ssen von den Mappern automatisch gel�scht
		werden. Das dies auch wirklich geschieht, wird durch Constraints auf der Datenbank 
		�berpr�ft.
	
		\subsection{Person Test}
			Es werden drei neue Personen Objekte erstellt. Zu jeder Person werden zwei 
			Adressen erstellt. Diese werden in die Datenbank geschrieben.
			
			Danach werden die Personen Objekte ge�ndert, je eine zus�tzliche Adresse 
			hinzugef�gt und ebenfalls in die Datenbank geschrieben.
		
		\subsection{Product Test}
			Es werden 16 Produkte Objekte in vier Kategorien erstellt und auf die Datenbank 
			geschrieben.
		
			Davon werden 12 Objekte ver�ndert und 4 Objekte wieder gel�scht. Diese Operationen 
			werden in einem Schritt auf der Datenbank ausgef�hrt.
			
		\subsection{Order Test}
			Die Personen und Produkte aus den vorherigen Tests werden �ber die Finder Methoden
			gesucht.
			
			Der ersten Person werden drei Bestellungen mit je drei Positionen hinzugef�gt. 
			Diese �nderungen werden auf die Datenbank geschrieben. Danach werden die Positionen 
			ge�ndert, hinzugef�gt und gel�scht.
	
	\section{Ergebnisse}
		Die Tests laufen ohne Fehler durch. Der Trace vom SQL Profiler und eine Kontrolle in 
		der Datenbank belegen, dass die Eintr�ge auch wirklich geschrieben wurden.
		
		Werden w�hrend den laufenden Tests die Datenbank-Eintr�ge ver�ndert, sollte eine Concurrency Exception geworfen werden. 
		Die Applikation st�rzt jedoch einer NullReferenzeException ab. Nach beheben dieses Fehlers reagiert alles wie erwartet.

