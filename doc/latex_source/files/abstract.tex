\begin{abstract}
\setcounter{page}{6}
\addcontentsline{toc}{chapter}{Abstract}
	\textit{STORM} ist ein Framework das es erm�glicht, Code f�r das O/R Mapping
	zu generieren. Das Framework besteht einerseits aus generischen Klassen, die
	von den selbst geschriebenen Klassen benutzt werden k�nnen und andererseits
	aus Templates, die benutzerdefinierte Klassen erzeugen. 
	Die Kernidee von \textit{STORM} ist, dass der Entwickler abstrakte Klassen
	schreibt, in denen er das Mapping zwischen Klassen der Applikation und
	Tabellen der Datenbank angeben kann. Diese abstrakten Klassen k�nnen
	anhand von Attributen, die in \textit{STORM} definiert wurden, parametriert werden.
	Diese Attribute und deren Werte werden zur Zeit der Code Generierung gelesen
	und dienen als Input der Templates. Der Output dieser Templates sind
	die auf das Problem angepassten Domain Objekt und Mapper Implementationen.
	
	Der generierte Code �bernimmt Aufgaben wie Lazy Load, Optimistic Offline Locking,
	Verwaltung und Ausf�hrung von Insert, Update und Delete sowie das dynamische Erstellen 
	von Abfragen mittels eines Query Objects. Zudem stellt er eine Factory zur Verf�gung, 
	um Instanzen der generierten Klassen erstellen zu k�nnen.
	
	Um den Code zu generieren wurde das Open Source Tool CodeSmith eingesetzt.
\end{abstract}