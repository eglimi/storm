\chapter{Conclusion}
	\section{Specific}
	A generative programming technique can be very powerful if used in the right context.
	Even though, it has its limitations, at least where productivity
	is concerned. One important thing is to precisely
	define the scope and the target when developing a tool such as \textit{STORM}.
	That is, it is important to know what the application which should be generated at the
	end needs to do. For example, to write templates for \textit{STORM} that can be used in 
	\emph{any} project which uses a database is nearly impossible. At least it would lack the
	whole benefit of code generation. Namely these benefits are that an application
	can be build in a much shorter time and the resulting application is error-free.
	But if every possible requirement had to be covered by the template, it would
	likely take a longer time to develop it than it ever can save.
	
	\section{In General}
	In general, a generative programming technique is useful whenever the 
	requirements for a project can be accurately specified from the beginning
	and do not change very much. Furhtermore, the project needs to be of
	a certain extend because it is doubtful if its worth the effort to create
	a framework for code generation just for a small application.
	
	The conclusion therefore is that code generation can help very much to shorten
	development time and to improve code quality. This, of course, is not limited
	to an object/relational mapper.
	
	\section{Outlook}
	\textit{STORM} could be extended in many ways. One important thing would be
	to implement different locking mechanism, another to support all database
	mapping types. Another extend would be to generate code for web service facade
	to domain object mappings.
	Although \textit{STORM} does not support this, it should give a reasonable start 
	for further development. 