\chapter{Vision}

	\section{Ziel der Arbeit}
	\label{sec:zielDerArbeit}
		Bei den meisten Applikationen gibt es eine bestimmte Anzahl von Codezeilen,
		die repetitiven Code darstellen. Das bedeutet, dieser Code muss bei jeder
		Applikation neu geschrieben werden. Ans�tze diese
		Arbeit auf ein Minimum zu reduzieren sind generative bzw. generische Programmiertechniken.
	
		Diese Arbeit soll ein Framework zur Verf�gung stellen, mit dem repetitiver Code
		generiert werden kann. Damit wird erreicht, dass die Software besser zu Warten 
		und der Entwicklungsprozess optimiert werden kann. Die Erkenntnis aus dieser
		Arbeit soll aus dem Bericht hervorgehen und soll zeigen, wie sinnvoll ein
		generativer Ansatz ist, um einen Objekt-Relationalen Mapper zu entwickeln.
		
		Das entwickelte Framework soll m�glichst einfach zu benutzen sein. Ein wichtiger
		Aspekt der Arbeit ist dabei die Erarbeitung eines geeigneten Konzeptes f�r die
		Parametrierung der Code Generierung, damit diese m�glichst generell eingesetzt
		werden kann.
	
	\section{Positionierung}

\chapter{Detaillierte Anforderungen}
		Im Folgenden werden die Anforderungen an die Software definiert. Nicht definiert
		werden hier die �brigen Ziele, die in der Vision (\ref{sec:zielDerArbeit})
		genannt wurden.
	
		\subsection{Muss-Ziele}
			
			\begin{enumerate}
				\item Eine L�sung f�r die Konfiguration der Code Generierung muss gefunden werden.
				\item Ben�tigte Templates f�r das Tool CodeSmith
							\cite{CodeSmith} werden erstellt.
				\item Eine L�sung f�r die Integration in die Visual Studio .NET
							Umgebung wird gefunden.
				\item	Falls ben�tigt, wird ein Plugin f�r CodeSmith geschrieben.
				\item Eine �berarbeitete Version der HsrOrderApp\_S3 wird erstellt
							(mit den erstellten CodeSmith Templates) und dokumentiert. Daraus
							entsteht die HsrOrderApp\_S4.
			\end{enumerate}
		
		\subsection{Optionale Ziele}
			
			\begin{enumerate}
				\item Eine zus�tzliche Beispielapplikation zur Veranschaulichung des Nutzen der Arbeit.
				\item	Ein Wizard f�r das Visual Studio .NET, um das Erstellen der Konfigurations-Dateien
							zu erleichtern.
			\end{enumerate}
			
		Da bei dieser Arbeit ein entscheidender Teil ist, ein L�sung f�r ein Problem zu finden, k�nnen
		auch keine spezifischen, sondern nur allgemein formulierte Anforderungen aufgef�hrt werden.
		Das Problem ist, dass wir noch nicht wissen, wie diese L�sung aussehen wird und deshalb auch
		keine Anforderungen an diese L�sung stellen k�nnen. Dies entspricht auch der Aufgabenstellung.