\chapter{Requirements Overview}

	\section{\textit{STORM} external}
		The main advantage of using a tool like \textit{STORM} is the time savings in
		project developments and the accuracy of the generated code. This in mind, the
		following requirements are specified:
		
		\begin{itemize}
			\item Mapping between the in-memory objects and the database tables is done automatically.
			\item Easy and understandable configuration and usage.
			\item There are no other dependencies than to CodeSmith, such that \textit{STORM} can
						be used independently.
			\item Is integrated in the make process of the Visual Studio .NET.
			\item Can also be used without Visual Studio .NET.
			\item The resulting program is efficient regarding network performance.
		\end{itemize}
		
		This is one would expect from this framework.
		Because it is not always easy to write templates for a general purpose,
		it is important to keep the target in mind. A framework like \textit{STORM}
		cannot cover all requirements which can occur in any project. An exactly
		defined boundary is needed. This boundary defines which problems can be solved
		with \textit{STORM}. It is described in the next section and in more
		detail in later chapters. The above requirements are of general nature and
		apply to every project which uses the technique of generative programming in
		the scope of an O/R Mapper.
	
	\section{\textit{STORM} internal}
		These requirements are more specific and could be called ``features of
		\textit{STORM}''. They are not directly connected with the usage of the 
		framework but affect the design of it. This list of features could
		be extended in many ways but it is a good starting point:
		
		\begin{itemize}
			\item No dependencies between the framework, the generated code and the code which
						uses them (namely a client) may exist.
			\item A locking mechanism is implemented.
			\item Insert, update and delete statements are handled by the framework and
						executed in the right order.
			\item SQL code is hidden from the user and generated automatically.
			\item Custom finder methods can be defined.
			\item Custom constructors can be defined.
			\item Transferred objects are stored to minimize remote calls.
			\item Data integrity is ensured.
			\item A well defined set of database mapping types are supported.
		\end{itemize}
		
		Most of these requirements are not easy to implement but even though, they are very
		important in terms of the usability of the framework. These requirements are
		a starting point for the whole design. Every one will be discussed in detail
		in this document.
		
		Because most of the above requirements can be solved by using and applying 
		appropriate patterns, the next section is devoted to these patterns.