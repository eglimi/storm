\chapter{Pers�nliche Berichte}

	\section{Michael Egli}
		Im Nachhinein kann muss ich sagen, dass dieses Projekt in vielen Hinsichten sehr spannend war.
		Zu Beginn war ich zwar etwas skeptisch, das hat sich aber schon bald in Interesse f�r das
		Projekt gewandelt. Das Spannende am Projekt f�r mich war, dass es sehr viele technische Aspekte
		tangierte. Zum Einen war dies XML und XML Schema, zum Anderen aber auch generische
		und generative Programmierung. Ebenfalls spannend war die pattern-orientierte Entwicklung.
		F�r mich war es das erste Projekt, bei dem ich so intensiv Patterns verwendet habe.
		
		Leider muss man sagen, dass wir sehr viele Stunden mit CodeSmith verbracht haben. Das Tool
		ist sehr schlecht dokumentiert und es funktioniert auch nicht alles, wie wir es erwartet
		haben.
		
		F�r mich ist das Resultat der Arbeit sehr erfreulich. Ich glaube wir k�nnen mit
		\textit{STORM} einen guten L�sungsansatz f�r das gestellte Problem pr�sentieren.
		Nat�rlich w�re es ausbauf�hig, aber die Zeit war leider sehr knapp daf�r.
		
		Das Projekt war �ber das Ganze gesehen sehr denklastig, was ich als positiv bewerte. Wir
		mussten uns sehr oft genau �berlegen, wie wir ein Problem l�sen k�nnen.
		
		Die Zusammenarbeit mit Marc kannte ich schon aus der letzten Semesterarbeit. Obwohl die
		Zeit w�hrend der Diplomarbeit sehr anstrengend war, hat die Teamarbeit immer funktioniert.
		Ich denke, dass wir uns sehr gut erg�nzt haben und auch, dass wir es geschafft haben die
		St�rken von beiden gut auszunutzen.
		
		Die Betreuung fand ich sehr gut. Zum Einen hatten wir Betreuung von Herrn Huser, zum
		Anderen auch noch von M. Konrad. Dies hat mich positiv �berrascht und hat uns auch 
		sehr oft geholfen. Es w�re sch�n, wenn es immer so w�re, aber leider haben wir 
		in fr�heren Projekten auch schon Anderes erlebt.
		
		Zusammenfassend kann ich sagen, dass ich sehr viel gelernt habe. Vor allem auch Dinge
		wie generative Programmierung, die ich unter Umst�nden in Zukunft gut gebrauchen kann.
		
		
	\section{Marc Winiger}
		Kurz vor Beginn der Diplomarbeit wurde das urspr�nglich geplante Thema verworfen. 
		Das neue Thema war generative Programmierung. Zuerst war ich eher skeptisch, denn 
		bisher habe ich immer wenn ich etwas Entwickelt habe, versucht das Problem so 
		generisch wie M�glich zu l�sen. Die generative Programmierung war f�r mich also 
		ein ganz neuer Aspekt.
		
		Es ist nicht einfach sich in Code einzuarbeiten, der von jemand anderem entwickelt 
		wurde. Das Ziel war es die existierende HsrOrderApp umzubauen und Teile davon mit 
		generiertem Code zu ersetzen. Beim �ndern einer existierenden Applikation, kann nach 
		unseren Erfahrungen sehr viel Zeit verloren gehen. Nach den negativen Erfahrungen der 
		letzen Semesterarbeit, bin ich froh, dass wir die M�glichkeit hatten, ein eigenes 
		Projekt zu starten und zu einem sp�teren Zeitpunkt die noch fehlenden Teile von der 
		existierenden Applikation zu �bernehmen. So hatten wir relativ schnell einen 
		Funktionierenden Prototypen. 
		
		Im Laufe des Projekt habe ich immer mehr Gefallen an der gerativen Programmierung 
		gefunden. Denn es ist zwar ein grosser Aufwand ein Template zu schreiben, das alle 
		F�lle abdeckt und auch noch fehlerfreien Code erzeugt. Doch wenn man das erst mal 
		gemacht hat, ist es genial, wenn man nur noch eine einfache Beschreibung braucht aus 
		denen dann hunderte Zeilen Code generiert werden.
		
		Obwohl wir am Anfang mit dem L�sungsansatz XML Schema theoretisch etwas Zeit verloren 
		haben, habe ich es interessant gefunden verschiedene Wege zu sehen, wie die Generierung 
		des Codes konfiguriert werden kann.
		
		Nachdem ich bereits bei der letzen Semesterarbeit mit Michael zusammen gearbeitet habe, 
		war f�r mich klar, dass ich auch die Diplomarbeit mit ihm durchstehen m�chte. Es sind 
		acht Wochen, in denen man sehr viel Zeit miteinander verbringt. Trotzdem hatten wir nie 
		ernsthafte Auseinandersetzungen. Ich w�rde mit ihm sofort wieder ein Projekt angehen.
		
		Es waren acht doch ziemlich strenge Wochen auf die wir zur�ckblicken k�nnen. Doch da wir 
		von Anfang ein Ziel vor Augen hatten, war es immer m�glich, sich irgendwie zu motivieren. 
		Alles in allem muss ich sagen dass es eine sch�ne, interessante und auch lehrreiche Zeit 
		war.
		