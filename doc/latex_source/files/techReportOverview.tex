\chapter{Overview}

	\section{About this document}
		This document has been written during a diploma thesis. It is the main
		part of the whole documentation and describes the core of this work. It gives
		a project overview and discusses all technical aspects.

	\section{Introduction}
		During 2003, three different versions of an order application were developed.
		This applications represent different approaches to deal with 
		databases. The application itself is not very complicated. It manages
		persons including addresses who can make orders. The respective database and
		application designs are discussed later in this document.
		
		The purpose of this work originated from the third of those applications. That version
		manages database related tasks through an O/R Mapper. Implementing such a mapper
		can be very challenging and error-prone. This is where \textit{STORM} comes into play.
		The idea behind it is to have a framework which generates all mapping code automatically,
		based on input parameters. This document describes how this can be achieved and also
		explains the limitation of such an approach. As a reference implementation to prove
		the usability of \textit{STORM}, a fourth order application has been created.
		Throughout this document, all implementation examples are taken from this reference
		implementation. At chapter \ref{cha:createApplication} the reference application 
		is taken as a real world example to show how to develop an application with \textit{STORM}.
		
		
	\section{About O/R Mapping}
		Today, most new applications are developed by using an object oriented programming
		language like C++ or C\#. Nevertheless, most databases have a relational structure.
		Relational means that relations between objects are always be modelled by defining foreign
		keys in tables. This structural difference is known as ``type mismatch''. Because of that, a mechanism
		is needed that provides a way to work with databases in an easy as possible way. One
		possibility is to use datasets. Datasets have a similar structure to that of a relational 
		database and are built-in types in C\#.
		This method was implemented in the first version of the order application.
		Another possibility is to write code which manages all the mapping. Namely this means
		mapping classes to tables, variables to columns, etc. This has the advantage that
		a programmer can code in a pure object oriented manner an does need to pay less
		attention to the database. For him, database tasks are done automatically. Although most
		O/R mapper, including \textit{STORM}, have more features implemented, mapping the in-memory
		structure to the database structure is the most important task of an O/R mapper.
	
	%---------
	\section{How to start}
		At the end, an O/R mapper is nothing else than code that is responsible for
		certain tasks. There are several methods to create this code. A catchphrase in 
		the context of software engineering is ``generic programming''. Generic programming is 
		a technique aiming at writing programs as general as possible, 
		without sacrificing efficiency by doing overgeneralising. Another technique is called
		``generative programming''. Generative programming enables programs to be automatically
		constructed. There are more programming techniques like aspect-oriented or functional
		programming which are not taken into account here.
		
		Both, generic and generative programming have their strengths and weaknesses.
		The third order application made use of the generic programming technique. As
		a counterpart, this thesis should result in the knowledge of the usability
		of a generative programming technique for this specific problem.
		
		As stated before, a generative programming technique enables programs to be automatically
		constructed. This sounds promising, but before one can start writing a program
		which constructs another program, he needs to define a starting point. Transferred
		to the problem of O/R mapping, this means that one needs to define where to make
		the mapping definitions. Three scenarios would be sensible:
		\begin{itemize}
			\item The database design is the starting point. Out of this design, all domain objects
						and the mapping code are generated.
			\item The domain model is the starting point. Out of this, the database tables and the mapping
						code are generated.
			\item Both, the domain model and the database design are taken as starting point. Additionally, the mapping 
						needs to be defined. Out of this, the mapping code is generated.
		\end{itemize} 
		
		Which approach should be chosen depends on the requirements for a given project. Each
		has been realised in commercial and open source projects. In this thesis,
		we have chosen to use the second approach where the domain model should be taken as starting point.
		Although with this approach the database tables could be generated automatically, it is not
		implemented because we use the existing database from the previous order applications. This is
		not a problem, because the mapping can be specified in the domain objects and this works
		whether the database tables are generated or already exist. 
				
	