\chapter{Code Generation}
\label{cha:codeGeneration}

	%----------
	\section{Finding a Tool}
		Mainly two projects exist which are able to generate code on basis of templates. These are
		Workstate's Codify\footnote{\url{http://www.workstate.com/codify}} and 
		CodeSmith\footnote{\url{http://www.ericjsmith.net/codesmith/}}. Both projects were evaluated
		at the beginning of this thesis. CodeSmith has several advantages over Codify. First, it
		is open source and it seems that it has a bigger community. Moreover, it cannot only be
		integrated in Visual Studio .NET, it has a command line tool and it can be run 
		from the CodeSmith standalone program. Particularly the command line tool is
		useful for creating makefiles. Furthermore, CodeSmith has a much more powerful
		way to write custom templates. Therefore, the decision clearly felt on CodeSmith. 
		
	%----------
	\section{Procedure of Code Generation}
		To be able to customise CodeSmith templates,
		one needs parameters as input for them. Input parameters are in the case of \textit{STORM} the custom 
		attributes in the abstract classes. The procedure to actually generate the code is as follows:\\[2mm]
		
			\noindent
			\fbox{Parameterised, abstract classes}
				$\xrightarrow{\text{compile}}$ 
				\fbox{DLL}
				$\xrightarrow{\text{input}}$
				\fbox{Assembly Loader}\\[2mm]
				$\xrightarrow[\text{a given class}]{\text{read type of}}$
				\fbox{CodeSmith}
				$\xrightarrow[\text{of class}]{\text{read attributes}}$
				\fbox{Generated Code}\\[2mm]
				
		\noindent
		First, all abstract classes must be compiled. The resulting DLL is used as input for
		the \verb~AssemblyLoader~ which is part of \textit{STORM}. The \verb~AssemblyLoader~ loads 
		the assembly and searches for a specified class. If the \verb~AssemblyLoader~
		finds the specified class in the assembly, it returns the type definition for it to CodeSmith.
		CodeSmith in turn takes this type definition and reads its custom attributes. On the basis of
		these attributes, the template code is executed and the resulting code is generated.
		
	%----------
	\section{Running the code}
		When all concrete classes have been generated, the code can be executed. To execute the code,
		the first step is to initialize the \verb~Registry~. This registry takes the runnable
		code as argument and searches for generated classes in it. Every class which is found
		is registered in a hashtable. Furthermore, the \verb~Registry~ operates as a
		\textit{Factory Method}. The procedure looks as follows:\\[2mm]
		
			\noindent
			\fbox{Executable}
				$\xrightarrow{\text{run}}$ 
				\fbox{Runnable code}
				$\xrightarrow{\text{call init}}$
				\fbox{Registry}\\[2mm]
				$\xrightarrow[\text{and register them}]{\text{search generated classes}}$
				\fbox{Execute operations}\\[2mm]
				