%%Pakete auswaehlen
\usepackage[ngerman,english]{babel}
%\usepackage[T1]{fontenc}
\usepackage[latin1]{inputenc}
\usepackage{fancyhdr}
\usepackage[pdftex]{graphicx}
\usepackage{color}
	\definecolor{darkblue}{rgb}{0.2,0.2,0.8}
\usepackage[pdftex, hyperfootnotes=false]{hyperref}
\usepackage{url}
\usepackage{here}
\usepackage{varioref}
\usepackage{rotating}
\usepackage{lscape}
\usepackage{pdfpages}
\usepackage{multirow}
\usepackage[nottoc]{tocbibind}  %literaturverzeichnis und index ins inhaltsverzeichnis
\usepackage{ltxtable}
\usepackage{latexsym}
\usepackage{amsmath}
\usepackage[headinclude,DIVcalc,BCOR5mm]{typearea}
\usepackage[hang,small,bf,longtable]{caption2}
\usepackage{fancyvrb}
\fvset{fontsize=\scriptsize}
\usepackage{listings}
\lstset{language=XML,tagstyle=\color[rgb]{0.5,0.2,0.5}}
\lstset{language=[CST]C,
        basicstyle=\tt\scriptsize,         
        keywordstyle=\color{blue}\bfseries,  
        identifierstyle=,
        commentstyle=\color[rgb]{0.5,0.2,0.5},
        stringstyle=\color[gray]{0.5},               
        showstringspaces=false,
        frame=lines,
        tabsize=2,
        numbers=left,
        numberstyle={\tiny},
        stepnumber=2
        }

%%index erstellen
%\usepackage{makeidx}
\makeindex

\bibliographystyle{is-plain}

\pagestyle{fancy}

\fancypagestyle{plain} {
\fancyfoot{}
\fancyhead{}
\renewcommand{\headrulewidth}{0pt}
}
\renewcommand{\chaptermark}[1]{%
\markboth{\chaptername\ \thechapter{}: #1}{}}
\renewcommand{\sectionmark}[1]{%
\markright{\thesection{}: #1}{}}

%%twoside header options
\fancyfoot{}% Unten nichts
\fancyhead[RE]{\itshape\leftmark}% Rechts auf geraden Seiten=innen
\fancyhead[LO]{\itshape\rightmark}% Links auf ungeraden Seiten=innen
\fancyhead[RO,LE]{\thepage}%Rechts auf ungeraden und links auf geraden Seiten

%%oneside header options - display only chapters
%\fancyfoot{}
%\lhead{\thepage}
%\rhead{\itshape\leftmark}

%% Color settings
\pagecolor{white} %background
\color{black}     % text

%% Attribute f�r das PDF-Dokument
\hypersetup{
	pdftitle = {Diploma Thesis},
	pdfsubject = {Diploma Thesis},
	pdfkeywords = {},
	pdfauthor = {Michael Egli, michael.egli@hsr.ch, Marc Winiger, marc.winiger@hsr.ch},
	colorlinks=true, 
	linkcolor=darkblue, 
	citecolor=darkblue, 
	urlcolor=darkblue, 
	plainpages=false
}

\newcommand{\highlight}[1]{\emph{#1}}
\newcommand{\command}[1]{\texttt{#1}}

%%additional options

%% Einzug verkleinern
\setlength{\parindent}{10pt} 

%set depth of toc to 2
\setcounter{tocdepth}{1}

%text on the part page
\makeatletter
\def\@endpart{}
\makeatother
