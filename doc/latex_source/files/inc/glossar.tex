%Bitte beim eintragen alphabetisch ordnen
\chapter{Glossary}

\begin{description}

\item[\Large{A}]
	\hypertarget{assembly}{}
	\item[assembly] A collection of one or more files that are versioned and deployed as a unit. 
									An assembly is the primary building block of a .NET Framework application. 
									All managed types and resources are contained within an assembly and are 
									marked either as accessible only within the assembly or as accessible from 
									code in other assemblies.

	\item[attribute] A .NET class used to describe code elements. This attributes can be read by reflection on runtime.

\item[\Large{C}]
	\item[CodeSmith] A freeware template-based code generator.

\item[\Large{D}]
	\item[DLL] ``Dynamic Link Library'' in .NET is an \hyperlink{assembly}{assembly}.

\item[\Large{G}]
	\item[GAC] ``Global Assembly Cache'' is the place where .NET Assembliesy can be registred. The 
	           Assembly needs to be signed. It is possible to register different Versions of the 
	           same Assembly.

\item[\Large{I}]
	\item[IDE] ``Integrated Developing Environment'' is a tool for developers. Such as Visual Studio .NET.

\item[\Large{M}]
	\item[makefile] File with a description of a build process. It defines dependencies between several tasks. Nmake is the .NET tool to execute the file.
	\item[mapper] An object-relational mapper maps the data from an object to a table on the database and vice versa.
	
\item[\Large{P}]
	\item[primary key] The primary key of a relational table uniquely identifies each record in the table. 
	It can either be a normal attribute that is guaranteed to be unique or it can be generated by the database.

\item[\Large{R}]
	\item[reflection] A mechanism to get information about running code.

\item[\Large{T}]
	\item[transaction] An inseparable list of database operations which must be executed either in its entirety or not at all.
	
\item[\Large{X}]
	\item[XML Schema Definition (XSD)] specifies how to formally describe the elements in an XML document. This description can be used to verify that each item of content in a document adheres to the description of the element in which the content is to be placed.

\end{description}
